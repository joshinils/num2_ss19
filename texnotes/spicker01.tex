\begin{minipage}[tc]{\linewidth}
	Lineares Ausgleichsproblem:
	\begin{tabular}{cl}
		\begin{tabular}{r|c|c|c|c}
			$t_i$ & 0 & 1 & 2 & 3 \\ 
			\hline 
			$y_i$ & 3 & $2,14$ & 1,8 & 1,72 \\ 
		\end{tabular}  & Daten \\ 
		func & Modellfunktion \\ 
	\end{tabular} 
\end{minipage}

\begin{minipage}[tc]{\linewidth}
	Satz: 1.1: \linebreak[3]
	$x^*\in \R$ ist genau dann eine Lösung des linearen Ausgleichsproblems, 
	wenn $x^*$ Lösung der Normalgleichung $A^TAx = A^Tb$ ist.
	Es gibt mindestens eine Lösung $x^*$.
	Sie ist eindeutig, gdw. $Rang(A) = n$.
\end{minipage}

