\documentclass[11pt]{article}

\usepackage{amsmath,amssymb, a4, verbatim}
\usepackage[german]{babel}
%\usepackage[latin1]{inputenc}
\usepackage[utf8]{inputenc} % üöäß
\usepackage{listings} % für inline codelistings
\lstset{%
	basicstyle=\ttfamily,	% the size of the fonts 
	columns=fixed,			% anything else is horrifying
	showspaces=false,	    % show spaces using underscores?
	showstringspaces=false,	% underline spaces within strings?
	showtabs=false,			% show tabs within strings?
	xleftmargin=1.5em,		% left margin space
}
\lstdefinestyle{inline}{basicstyle=\ttfamily}
\newcommand{\listline}[1]{\lstinline[style=inline]!#1!}


\usepackage{caption}
\newcommand{\tinycaption}[1]{\captionsetup{labelformat=empty}\caption{#1}}

%\usepackage{color}
%\usepackage{epsfig} % eps
\usepackage{graphicx} % eps
%\usepackage[shortcuts]{extdash}
%\usepackage{dsfont}
%\usepackage{epstopdf} % eps
%\usepackage[pdf]{pstricks} % eps
%\usepackage{auto-pst-pdf}
\usepackage{mathtools}
\usepackage{dsfont} % $ \mathds{1} $
%\usepackage{icomma}
\usepackage{tikz}
%\usepackage{pgfplots}
%\pgfplotsset{compat=1.8}
\usepackage[bottom]{footmisc} % put footnotes at the bottom of page
\usepackage{nicefrac} % für brüche die aussehen wie prozentzeichen
% \usepackage{ps2pdf}
\usetikzlibrary{automata,positioning}

\usepackage{algorithmicx}
\usepackage{algpseudocode}
\usepackage{algorithm}

\usepackage{multicol}
\usepackage{wrapfig} % make stuff float
\usepackage{placeins} % stop stuff from floating
\usepackage{seqsplit} % very long numbers
\usepackage{framed} % begin{framed}

%  Headings and Footings :
\usepackage{fancyhdr}
\headheight15pt
\lhead{Numerik II, \ueberschrift}

\chead{}
\rhead{\thepage}
\renewcommand{\headrulewidth}{.4pt}

\lfoot{\today}
\cfoot{}
\rfoot{Joshua}
\renewcommand{\footrulewidth}{.4pt}

%----------------------------------------------------------------

\textwidth16.5cm
\oddsidemargin0.cm
\evensidemargin0.cm

\def\somedistanceTop{3cm}
\def\somedistanceLeft{2cm}

\usepackage{geometry}
\geometry{
%	a4paper,
%	total={170mm,257mm},
	left=\somedistanceLeft,
	right=\somedistanceLeft,
	top=\somedistanceTop,
	bottom=\somedistanceTop
}


\parindent0cm

\usepackage{cancel}

\newcommand{\R}{ {\mathbb R} }
\newcommand{\C}{ {\mathbb C} }
\newcommand{\1}{ {\mathds{1}} }
\newcommand{\abs}[1]{\lvert#1\rvert}
\newcommand{\norm}[1]{\left\lVert#1\right\rVert}
\newcommand{\xt}{\tilde{x}}
\newcommand{\dotleq}{\dot{\leq}}
\newcommand{\m}{\hphantom{-} }

\newcommand{\dashfill}[1]{\vspace{11pt}\def\dashfill{\cleaders\hbox{#1}\hfill}\hbox to \hsize{\dashfill\hfil}\vspace{11pt}}
\newcommand{\scdot}{\!\cdot\!}


\newcommand{\sig}{\text{signum}}
\newcommand{\rot}{}

% ------------------  edit Ueberschrift ---------------------
\newcommand{\ueberschrift}{Numerik II Notizen}


\usepackage{mathpazo}
\usepackage[mathpazo]{flexisym}
\usepackage{breqn}

% https://tex.stackexchange.com/questions/110393/too-wide-figure-caption
% ffs \usepackage{boxhandler}

\newcommand{\scalprod}[2]{\left\langle#1;#2\right\rangle}

\usepackage{url}

\usepackage{pdfpages}
%\usepackage{showframe}
% -----------------------------------------------------------


%\texttt{} num.num.alph as headings
\renewcommand{\thesubsubsection}{\thesubsection.\alph{subsubsection}}

\begin{document}
	\pagestyle{fancy}
	
	ue1:

1.Aufgabe:
\begin{equation}
	\begin{split}
		b \in \left\{ x_1 \cdot \begin{pmatrix*} 1\\ 1\\ 1 \end{pmatrix*} + x_2 \cdot \begin{pmatrix*} 1\\ -2\\ -1 \end{pmatrix*} | x_1, x_2 \in \R \right\}
	\end{split}
\end{equation}

	\section*{2. Übung}
\subsection*{2.1. Aufgabe}
Zu zeigen:
\begin{equation}\begin{split}
	A^TAx = 0 \Leftrightarrow Ax = 0
\end{split}\end{equation}


\subsection*{2.2. Aufgabe}
\begin{equation}\begin{split}
	A &= \begin{pmatrix}
		-1 & 1\\
		2 & 4\\
		-2 & -1
	\end{pmatrix}\\
	&= Q\cdot R\\
	&\stackrel{\text{\lstinline|Matlab|}}{=}
	\underbrace{
		\nicefrac{1}{3}\begin{pmatrix}
		-1&2&-2\\
		2&2&1\\
		-2&1&2
	\end{pmatrix}}_{Q}
	\cdot
	\begin{pmatrix*}
		3&3\\
		0&3\\
		0&0
	\end{pmatrix*}
\end{split}\end{equation}

Minimale 2-Norm des Residuums?
\begin{equation}\begin{split}
	\norm{b-Ax}_2^2 &= \norm{\underbrace{Q^Tb}_{c} -Rx}_2^2\\
	&=
	\norm{\begin{pmatrix}
			c_1\\
			c_2
		\end{pmatrix}
		-\begin{pmatrix}
			R_1\\
			\vec{0}
		\end{pmatrix}
		\scdot x
	}_2^2 \\
	&= \norm{c_1 -R_1x}_2^2 + \norm{c_2 -\vec{0}x}\\
\end{split}\end{equation}

nun könnte für $\norm{c_1 -R_1x}_2^2$ ein x gefunden werden, da aber $\norm{c_2 \quad\cancelto{0}{-\vec{0}x}\quad} \geq 0$ ist, kann das Residuum nur minimal $\norm{c_2}_2^2$ werden.

Hier also:
\begin{equation}\begin{split}
	Q^Tb &= \begin{pmatrix}
		-1\\2\\1
	\end{pmatrix}\\
	c_2 &= 1 \Rightarrow
	\norm{c_2}_2^2 = 1
\end{split}\end{equation}



\subsection*{2.3. Aufgabe}
\begin{equation}\begin{split}
	eq
\end{split}\end{equation}


\end{document}














































