\section*{2. Übung}
\subsection*{2.1. Aufgabe}
Zu zeigen:
\begin{equation}\begin{split}
	A^TAx = 0 \Leftrightarrow Ax = 0
\end{split}\end{equation}


\subsection*{2.2. Aufgabe}
\begin{equation}\begin{split}
	A &= \begin{pmatrix}
		-1 & 1\\
		2 & 4\\
		-2 & -1
	\end{pmatrix}\\
	&= Q\cdot R\\
	&\stackrel{\text{\lstinline|Matlab|}}{=}
	\underbrace{
		\nicefrac{1}{3}\begin{pmatrix}
		-1&2&-2\\
		2&2&1\\
		-2&1&2
	\end{pmatrix}}_{Q}
	\cdot
	\begin{pmatrix*}
		3&3\\
		0&3\\
		0&0
	\end{pmatrix*}
\end{split}\end{equation}

Minimale 2-Norm des Residuums?
\begin{equation}\begin{split}
	\norm{b-Ax}_2^2 &= \norm{\underbrace{Q^Tb}_{c} -Rx}_2^2\\
	&=
	\norm{\begin{pmatrix}
			c_1\\
			c_2
		\end{pmatrix}
		-\begin{pmatrix}
			R_1\\
			\vec{0}
		\end{pmatrix}
		\scdot x
	}_2^2 \\
	&= \norm{c_1 -R_1x}_2^2 + \norm{c_2 -\vec{0}x}\\
\end{split}\end{equation}

nun könnte für $\norm{c_1 -R_1x}_2^2$ ein x gefunden werden, da aber $\norm{c_2 \quad\cancelto{0}{-\vec{0}x}\quad} \geq 0$ ist, kann das Residuum nur minimal $\norm{c_2}_2^2$ werden.

Hier also:
\begin{equation}\begin{split}
	Q^Tb &= \begin{pmatrix}
		-1\\2\\1
	\end{pmatrix}\\
	c_2 &= 1 \Rightarrow
	\norm{c_2}_2^2 = 1
\end{split}\end{equation}



\subsection*{2.3. Aufgabe}
\begin{equation}\begin{split}
	eq
\end{split}\end{equation}
