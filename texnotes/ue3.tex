\section*{SVD}
\begin{equation}\begin{split}
    A &= U \Sigma V\\
    A\cdot A^T u &= \lambda u \\
    A^T \cdot A v &= \lambda v
\end{split}\end{equation}
Die ersten $p$ Eigenwerte von $A$, $A^TA$ und $AA^T$ sind identisch.

Die $i$-ten Eigenvektoren von $A^TA$ und respektive $AA^T$ sind die $i$-ten Spalten von $U$ und $V$.

Die Diagonale von $\Sigma$ sind $\sigma_{ii} = \sigma_{i} = \sqrt{\lambda_i}$

\section*{3. Übung}
\subsection*{3.1.i Aufgabe}
\begin{equation}\begin{split}
    AV = VD \\
    \Rightarrow
    A \cdot v_i = \lambda_i \cdot v_i\\
    \Leftrightarrow 
    A \cdot v_i = v_i \cdot \lambda_i \\
\end{split}\end{equation}
d.h.

\subsection*{3.1.ii Aufgabe}
Zu zeigen: für EW $v_i$ von $A$
\begin{equation}\begin{split}
	A^T = A &\Rightarrow v_i \bot v_j \quad\forall i \neq j\\
	&\Leftrightarrow \scalprod{v_i}{v_j} = \vec{0}
\end{split}\end{equation}
Tipp: Berechnen Sie zu zwei Eigenvektoren $v_1$ und $v_2$ das Skalarprodukt $\scalprod{v_1}{Av_2}$ und benutzen Sie die Identität $\scalprod{x}{Ax} = \scalprod{A^Tx}{y}$.

\begin{equation}\begin{split}
	...
\end{split}\end{equation}

\subsection*{3.4.a Aufgabe}

\begin{equation}\begin{split}
    A 
    &= 
    \begin{pmatrix}
    	1&1\\
    	1&1\\
    \end{pmatrix}\\
    A^TA 
    &= 
	\begin{pmatrix}
		2&2\\
		2&2\\
	\end{pmatrix}\\
	&\Rightarrow \lambda_1 = 0, \lambda_2 = 4\\
    &\Rightarrow v_1 = \begin{pmatrix}-1\\1\end{pmatrix}, v_2 = \begin{pmatrix}1\\1\end{pmatrix}\\
    \text{Sortiere: ... }&\leadsto\\
    AA^T 
    &= 
	\begin{pmatrix}
		2&2\\
		2&2\\
	\end{pmatrix} \quad analog\\
	&\Rightarrow
	U = V 
	=
	\begin{pmatrix}
		\sqrt{2}&-\sqrt{2}\\
		 \sqrt{2}&\sqrt{2}\\
	\end{pmatrix}\\
	&D 
	=
	 \begin{pmatrix}
		\sqrt{\lambda_1}& 0 \\
		0 & \sqrt{\lambda_2}
	\end{pmatrix}
	=
	\begin{pmatrix}
		0&0\\
		0&2
	\end{pmatrix}\\
	\Rightarrow
	A 
	&= 
	\begin{pmatrix}
		\sqrt{2}&-\sqrt{2}\\
		\sqrt{2}&\sqrt{2}\\
	\end{pmatrix}
	\cdot
	\begin{pmatrix}
		2&0\\
		0&0
	\end{pmatrix}
	\cdot 
	\begin{pmatrix}
		\sqrt{2}&-\sqrt{2}\\
		\sqrt{2}&\sqrt{2}\\
	\end{pmatrix}^T\\
\end{split}\end{equation}

\begin{equation}\begin{split}
	b &= \begin{pmatrix}3\\3\end{pmatrix}\\
	\leadsto 
	Ax &= b\\
	\Leftrightarrow
	U \Sigma V^T  x &= b\\
	\Leftrightarrow
	\Sigma V^T x &= U^T b\\
	\Leftrightarrow
	V^T x &= \Sigma^+ U^T b\\
	x &= V \Sigma^+ U^T b\\
	\Leftrightarrow
	x &=
	\begin{pmatrix}
		\sqrt{2}&-\sqrt{2}\\
		\sqrt{2}&\sqrt{2}\\
	\end{pmatrix}
	\cdot 
	\begin{pmatrix}
		\nicefrac{1}{2}&0\\
		0&0
	\end{pmatrix}
	\cdot	
	\begin{pmatrix}
		\sqrt{2}&-\sqrt{2}\\
		\sqrt{2}&\sqrt{2}\\
	\end{pmatrix}^T
	\cdot 
	\begin{pmatrix}3\\3\end{pmatrix}\\
	x &\stackrel{Matlab}{=}
	\begin{pmatrix}
		\nicefrac{1}{3}\\\nicefrac{1}{3}
	\end{pmatrix}
\end{split}\end{equation}
oder alternativ einfacher:

\begin{equation}\begin{split}
    A\in \R^{n\times n} \text{regulär}, Ax &= b\\
    x = A^{-1} b &= V\Sigma^{-1}U^T\cdot b\\
    &=\sum_{i=1}^{p=r}\frac{1}{\sigma_i} v_i \cdot \left(u_i^Tb\right)\\
    \Rightarrow 
    x = 
    \nicefrac{1}{2}
    \begin{pmatrix}
	    \sqrt{2}\\\sqrt{2}
    \end{pmatrix}
    \cdot 
    \left(
   		\begin{pmatrix}
   			\sqrt{2}\\\sqrt{2}
   		\end{pmatrix}^T
   		\begin{pmatrix}
	   		3\\3
   		\end{pmatrix}
	\right)
\end{split}\end{equation}


\subsection*{3.4.b Aufgabe}

\begin{equation}\begin{split}
    A &= \begin{pmatrix}
	    -1&1\\
	    2&0\\
	    1&0
    \end{pmatrix}\\
    AA^T
    &=
    \begin{pmatrix}
	    2&-2&-1\\
	    -2&4&2\\
	    -1&2&1
    \end{pmatrix}\\
    A^TA
    &=
    \begin{pmatrix}
	    6&-1\\
	    -1&1
    \end{pmatrix}\\
    \det\left(AA^T-\lambda \1\right)
    &=
    (6-\lambda)(1-\lambda) -1 \stackrel{!}{=} 0\\
    \Leftrightarrow
    \lambda^2 -7\lambda + 5 &= 0\\
    \Rightarrow
    \lambda_{1/2} 
    &=
    \nicefrac{7}{2} \pm \sqrt{\frac{7^2}{4} -5}
    =
	\nicefrac{7}{2} \pm \frac{\sqrt{29}}{2}
\end{split}\end{equation}

\begin{equation}\begin{split}
	\left(A^TA-\lambda\1\right)\underline{v_1} &\stackrel{!}{=} \underline{0}\\
	\Rightarrow
	\begin{pmatrix}
		6-\lambda_1&-1\\-1&1-\lambda_1
	\end{pmatrix}
	\begin{pmatrix}
		v_{11}\\v_{12}
	\end{pmatrix}
	&=
	\begin{pmatrix}
		0\\0
	\end{pmatrix}\\
	\Rightarrow
	\begin{pmatrix}
		6-\nicefrac{7}{2} - \frac{\sqrt{29}}{2} & -1\\
		-1 & 1 -\nicefrac{7}{2} - \frac{\sqrt{29}}{2}
	\end{pmatrix}
	v_1
	&=
	0\\
	\Rightarrow
	\begin{pmatrix}
		\nicefrac{5}{2} - \frac{\sqrt{29}}{2} & -1\\
		-1 & -\nicefrac{5}{2} - \frac{\sqrt{29}}{2}
	\end{pmatrix}
	v_1
	&=
	0\\
	\Rightarrow
	\begin{pmatrix}
		\nicefrac{5}{2} - \frac{\sqrt{29}}{2} & -1\\
		-1 \cdot (\nicefrac{5}{2} - \frac{\sqrt{29}}{2}) & (-\nicefrac{5}{2} - \frac{\sqrt{29}}{2})\cdot(\nicefrac{5}{2} - \frac{\sqrt{29}}{2})
	\end{pmatrix}
	v_1
	&=
	0\\
	\stackrel{\text{sum}}{\Rightarrow}
	(-\nicefrac{5}{2} - \frac{\sqrt{29}}{2})\cdot(\nicefrac{5}{2} - \frac{\sqrt{29}}{2}) \cdot v_{12} &= 0\\
	\Rightarrow v_{12} &= 0\\
	\Rightarrow v_1 &= \underline{0}
\end{split}\end{equation}
















